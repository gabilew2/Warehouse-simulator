\chapter{Warehouse-\/simulator}
\hypertarget{md__r_e_a_d_m_e}{}\label{md__r_e_a_d_m_e}\index{Warehouse-\/simulator@{Warehouse-\/simulator}}
\label{md__r_e_a_d_m_e_autotoc_md0}%
\Hypertarget{md__r_e_a_d_m_e_autotoc_md0}%
 Projekt Symulacja Domu Towarowego to system wspomagający zarządzanie magazynem, który umożliwia symulowanie i optymalizowanie procesów magazynowych. Dzięki niemu użytkownicy mogą modelować różne scenariusze, analizować wyniki oraz podejmować decyzje w czasie rzeczywistym. Projekt integruje algorytmy symulacyjne z interaktywnym interfejsem użytkownika, co pozwala na dynamiczne monitorowanie stanu magazynu oraz szybką reakcję na zmiany w otoczeniu biznesowym.

Główne cechy projektu obejmują definiowanie magazynów, produktów, atrybutów, symulację działania magazynu w pętli czasowej, generowanie zdarzeń, interwencję użytkownika oraz generowanie raportów z wynikami symulacji. Celem projektu jest dostarczenie narzędzia wspomagającego efektywne zarządzanie magazynem, które pozwoli firmom na zwiększenie efektywności operacyjnej i maksymalizację zysków. 